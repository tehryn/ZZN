\documentclass[11pt,a4paper,titlepage]{article}
\usepackage[left=1.5cm,text={18cm, 25cm},top=2.5cm]{geometry}
\usepackage[utf8]{inputenc}
\usepackage{setspace}
\usepackage{graphicx}
\usepackage[czech]{babel}
\usepackage{float}
\usepackage{color}
\usepackage{hyperref}
% \bibliographystyle{czplain}
\usepackage{etoolbox}
% \apptocmd{\thebibliography}{\raggedright}{}{}
\setlength{\parindent}{0cm}
\setlength{\parskip}{1em}
\sloppy

\hypersetup{
	colorlinks=true,
	linktoc=all,
	linkcolor=blue,
	citecolor=red,
	urlcolor=blue,
}

\begin{document}

		\setstretch{0.5}
		\begin{center}

			\includegraphics[width = 150mm]{logo.png}\\

			\vspace{\stretch{0.382}}

			\LARGE
			Získávání znalostí z databází\\
			Databáze ze sčítání lidu\\
			\vspace{\stretch{0.618}}

		\end{center}

	\Large{\today} \hfill Jiří Matějka, Lucie Pelantová
	\thispagestyle{empty}
	\newpage
	\setcounter{page}{1}

    \section{Úvod}
        Cílem této práce je získání informacích ze zvolené datové sady. V případě tohoto projektu
        je zvolenou sadou dat databáze ze sčítání lidu.
    
    \section{Popis dat}
        Datová sada se skládá ze 3 souborů - 2 CSV soubory a 1 soubor popisující data (význam
        jednotlivých sloupců, k čemu data byla použita a jak byla zpracována). CSV soubory jsou sice 2,
        ale menší soubor obsahuje podmnožinu dat většího a proto se ve výsledku pracuje jen s jedním CSV souborem.
        
        \subsection{Popis jednotlivých sloupců}
            Soubor dat se skládá z celkem z 32561 záznamů a každý záznam z 15 sloupců. U některých záznamů mohou chybět údaje pro jeden či více sloupců.
            \begin{itemize}
                \item Age (věk) -- celé číslo udávající věk osoby v letech
                \item Workclass (pracovní třída) -- V jakém odvětví osoba pracuje. Může obsahovat následjící hodnoty: \textit{Private, Self-emp-not-inc, Self-emp-inc, Federal-gov, Local-gov, State-gov, Without-pay, Never-worked}.
                \item Final weight (finální hodnota) -- celé číslo, pomocí kterého lze určit jak moc si jednotlivé záznamy jsou podobné na základě demografických údajů (čím menší je rozdíl těchto hodnot, tím více si jsou záznamy podobné).
                \item Education (Vzdělání) -- vzdělání, kterého osoba dosáhla. Může obsahovat následující hodnoty: \textit{Bachelors, Some-college, 11th, HS-grad, Prof-school, Assoc-acdm, Assoc-voc, 9th, 7th-8th, 12th, Masters, 1st-4th, 10th, Doctorate, 5th-6th, Preschool}.
                \item Education number -- Číslo vzdělání??? Možná počet let, kolik studovali??? Číslo ukazující kvalitu vzdělání???
                \item Marital status (Rodinný stav) -- Rodinný stav osoby. Může obsahovat následující hodnoty: \textit{Married-civ-spouse, Divorced, Never-married, Separated, Widowed, Married-spouse-absent, Married-AF-spouse}.
                \item Occupation (Zaměstnání) -- Zaměstnání osoby. Může nabývat následujících hodnot: \textit{Tech-support, Craft-repair, Other-service, Sales, Exec-managerial, Prof-specialty, Handlers-cleaners, Machine-op-inspct, Adm-clerical, Farming-fishing, Transport-moving, Priv-house-serv, Protective-serv, Armed-Forces}.
                \item Relationship Vztah/role v rodině -- Může obsahovat následující hodnoty: \textit{Wife, Own-child, Husband, Not-in-family, Other-relative, Unmarried}.
                \item Race (Rasa) -- Rasová příslušnost danné osoby. Může obsahovat následjící hodnoty: \textit{Wife, Own-child, Husband, Not-in-family, Other-relative, Unmarried}
                \item Sex (Pohlaví) -- údaj, zda se jedná o muže či ženu.
                \item Capital gain (Kapitálový příjem) -- celé číslo udávající 
                \item Capital loss (Kapitálová ztráta) -- celé číslo udávající 
                \item Hours per week (Pracovní úvazek) -- celé číslo udávající počet hodin, které osoba odpracuje během jednoho týdne
                \item Native country (Rodná země) -- Země, ve které se osoba narodila
                \item Sallary (Mzda) -- Údaj, zda osoba vydělává více jak 50 000 (asi dolarů) ročně.
            \end{itemize}
    
    \section{Řešená úloha}    
    \begin{itemize}
        \item Pokud chci vydělávat >50k, s jakým vzděláním a zaměstnáním mám největší šanci toho dosáhnout? Jak tuto skutečnost ovlivní rodina, rasa a pohlaví?
        \item nějaké nápady????
    \end{itemize}
    % \tableofcontents
	% \newpage
	% \newpage
    % 
    % \section{Úvod}
	% \newpage
	% \bibliography{zdroje}

\end{document}
