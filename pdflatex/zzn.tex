\documentclass[11pt,a4paper,titlepage]{article}
\usepackage[left=1.5cm,text={18cm, 25cm},top=2.5cm]{geometry}
\usepackage[utf8]{inputenc}
%\usepackage{setspace}
\usepackage{graphicx}
\usepackage[czech]{babel}
\usepackage{float}
\usepackage{color}
\usepackage{hyperref}
% \bibliographystyle{czplain}
\usepackage{etoolbox}
% \apptocmd{\thebibliography}{\raggedright}{}{}
\setlength{\parindent}{0cm}
\setlength{\parskip}{1em}
\sloppy

\hypersetup{
	colorlinks=true,
	linktoc=all,
	linkcolor=blue,
	citecolor=red,
	urlcolor=blue,
}

\begin{document}

		%\setstretch{0.5}
		\begin{center}

			\includegraphics[width = 150mm]{logo.png}\\

			\vspace{\stretch{0.382}}

			\LARGE
			Získávání znalostí z databází\\
			Databáze ze sčítání lidu - zadání\\
			\vspace{\stretch{0.618}}

		\end{center}

	\Large{\today} \hfill Jiří Matějka, Lucie Pelantová
	\thispagestyle{empty}
	\newpage
	\setcounter{page}{1}

    \section{Úvod}
        Cílem tohoto projektu je získání zajímavých souvislostí z vybrané datové sady. Pro tento 			účel byla použita data ze sčítání lidu.
        
        Dolováním z této datové sady bude v rámci tohoto projektu zkoumán vliv zázemí občana na výběr jeho povolání. Vytyčené cíle tohoto projektu jsou blíže popsané v kapitole \ref{cile}.
    
    \section{Popis dat}
    Data ze sčítání lidu byla získána z databáze Census Bureau\footnote{http://www.census.gov/ftp/pub/DES/www/welcome.html}. Data extrahoval Berry Becker z databáze z roku 1994 a následně z nich pomocí sady podmínek (například omezení na věk občana)  vybral vypovýdající záznamy. Poprvé tyto data citoval Ron Kohavi v článku \textit{Scaling Up the Accuracy of Naive-Bayes Classifiers: a Decision-Tree Hybrid}.
    
        Datová sada se skládá ze 2 částí - soubor \textbf{adult.names} obsahuje popis jednotlivých údajů a soubor \textbf{adult.data} obsahuje samotná data.
    
    \section{Cíle projektu\label{cile}}
    
    Tento projekt si klade za cíle prozkoumání zajímavých souvislostí mezi zázemím občana a jeho volbou povolání. Při určování zázemí občana budou brány v potaz tyto údaje obsažené v datové sadě:
    
    \begin{itemize}
    	\item Věk
		\item Dosažené vzdělání
		\item Rodinný stav
		\item Role v rodině
		\item Rasová příslušnost
		\item Pohlaví
		\item Rodná země
    \end{itemize}
    
    
	Vliv na zvolené zaměstnání bude zkoumán ve dvou různých směrech:
    \begin{itemize}
        \item Pracovní třída - občan pracuje pod zaměstnavatelem, ve státním sektoru, jako osoba samostatně výdělečně činná, ...
        \item Povolání - občan pracuje například jako prodejce, poskytuje služby, v zemědělství, ...
    \end{itemize}
    % \tableofcontents
	% \newpage
	% \newpage
    % 
    % \section{Úvod}
	% \newpage
	% \bibliography{zdroje}

\end{document}
